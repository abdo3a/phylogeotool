\documentclass[a4paper, 11pt]{article} % Font size (can be 10pt, 11pt or 12pt) and paper size (remove a4paper for US letter paper)

\usepackage[protrusion=true,expansion=true]{microtype} % Better typography
\usepackage{graphicx} % Required for including pictures
\usepackage{hyperref}
\usepackage{float}

\usepackage{mathpazo} % Use the Palatino font
\usepackage[T1]{fontenc} % Required for accented characters
\linespread{1.05} % Change line spacing here, Palatino benefits from a slight increase by default

\makeatletter
\renewcommand\@biblabel[1]{\textbf{#1.}} % Change the square brackets for each bibliography item from '[1]' to '1.'
\renewcommand{\@listI}{\itemsep=0pt} % Reduce the space between items in the itemize and enumerate environments and the bibliography

\renewcommand{\maketitle}{ % Customize the title - do not edit title and author name here, see the TITLE block below
\begin{flushright} % Right align
{\LARGE\@title} % Increase the font size of the title

\vspace{50pt} % Some vertical space between the title and author name

{\large\@author} % Author name
\\\@date % Date

\vspace{40pt} % Some vertical space between the author block and abstract
\end{flushright}
}

%----------------------------------------------------------------------------------------
%	TITLE
%----------------------------------------------------------------------------------------

\title{\textbf{Phylogeotool}\\ % Title
Installation Reference Manual} % Subtitle

\author{\textsc{Ewout Vanden Eynden, Pieter Libin, Kristof Theys, anderen, Guy Baele} % Author
\\{\textit{Rega Institute for Medical Research, KU Leuven}}} % Institution

\date{August 2016} % Date

%----------------------------------------------------------------------------------------

\begin{document}
\maketitle % Print the title section

\vspace{30pt} % Some vertical space between the abstract and first section

%------------------------------------------------
\tableofcontents
\newpage

\section{Installation}

\subsection{Using pre-built JAR files}
If you don't want to build the JARs/WAR from source, you can skip the following sections and proceed to section \ref{sec:jars}.
%PL: is a building from source really in its place in this document?
%PL: I would just explain how to install the jars, and refer to a BUILD file that explains how to build the jars?


\subsection{Prerequisites for building your own}

\subsubsection*{Java}
Download and install the newest Java Development Kit (JDK) from \url{http://www.oracle.com/technetwork/java/javase/downloads/index.html}.
The current version of PhyloGeoTool was built using JDK 1.8.0\_31.
For example, on Ubuntu Linux, Oracle Java 8 can be installed as follows (although OpenJDK should work just fine):
\begin{verbatim} 
sudo add-apt-repository ppa:webupd8team/java
sudo apt-get update
sudo apt-get install oracle-java8-installer
\end{verbatim}

\subsubsection*{Tomcat}
%PL: this is not a prerequisite for building your own jars, this is confusing.
%PL: I would really focus on installing in this doc, not on Building.
Download and install the latest Tomcat version from \url{http://tomcat.apache.org}.
For example:
\begin{verbatim}
sudo apt-get update
sudo apt-get install tomcat7
\end{verbatim}

\subsubsection*{Github}
Download the code from \url{https://github.com/rega-cev/phylogeotool/}. 
The project is currently still private. 
During the trial period you can send an email to \href{mailto:phylogeotool@kuleuven.be}  {phylogeotool@kuleuven.be} with your Github account name to get read rights on the project.
Git is readily available on most operating systems; if not, it can be installed as follows:
\begin{verbatim}
sudo apt-get install git-all
\end{verbatim}

\subsubsection*{Ant}
Download and install the newest Ant version from \url{http://ant.apache.org/} as we will use it to build our project.
The current version of PhyloGeoTool was built using Ant 1.9.4.
Ant can be installed as follows:
\begin{verbatim}
sudo apt-get install ant
\end{verbatim}

\subsubsection*{Phylogenetic tree}
%PL: again, this is not a prerequisite for building your own jars, this is confusing.
A rooted binary phylogenetic tree in either Nexus or Newick format, which may or may not be time-stamped.

\subsubsection*{CSV file}
%PL: again, this is not a prerequisite for building your own jars, this is confusing.
A comma-separated value (CSV) file, containing an ID column that contains IDs that correspond to the taxa names of the provided phylogenetic tree.
Remaining columns may contain additional information/annotation for the taxa in the phylogenetic tree, such as geographic location, virus genotype/subtype, patient attributes (age, ethnicity), \ldots
%PL: since the geographic location is interpreted to build up the map, I would think the application needs to know which column is providing info for this. Is this done by using a fixed name? If so mention it here.

\subsection{Building the project}

\subsubsection{Ubuntu Linux}

We here outline the various steps necessary for a successful build of the project.
The build process here is described as it was performed on an Ubuntu 14.04 LTS (Trusty Tahr) installation.
\begin{itemize}
\item Create an empty directory
\item {In that directory, clone the git repository: 
\begin{verbatim}
git clone https://github.com/rega-cev/phylogeotool/
cd phylogeotool
ant
\end{verbatim}
}
\item This build process generates a phylogeotool-01.war in the dist directory. Copy this file to the webapps folder of Tomcat. This enables browser access to a localhost version of the PhyloGeoTool.
\end{itemize}

\section{Data preparation}
\label{sec:jars}
After the project has been built successfully, different jar files can be found in the dist folder. 
Their function is explained in the sections below.

\subsection{DistanceMatrix.jar}
%GB: Is this mandatory? Why do we need to do this? Would be nice to explain here.
This jar file contains a tool to create a distance matrix based on the phylogenetic tree that will be used in the PhyloGeoTool.
DistanceMatrix.jar uses one input file, a phylogenetic tree to be used in the PhyloGeoTool, and will output the distance matrix that is derived from the phylogenetic tree. %GB: what is this phylo.tree? What format should it be in? Should it have a specific extension?
For example, the following command uses DistanceMatrix.jar to generate a distance matrix from a previously reconstructed phylogenetic tree in Newick format: 
java -jar DistanceMatrix.jar tree.newick distances.csv
%is the distance matrix a CSV file?

\subsection{PreRender.jar}
%GB: Does this step somehow require the distance matrix computed in the previous section?
%PL: yes
A major goal of the PhyloGeoTool is to cluster the tree index the taxa annotations, of which the computation takes a long time and its runtime depends on the size of the phylogenetic tree. However, this computation can be performed before the installation of the web-tool, using PreRender.jar, allowing the web-tool to render both clusters and annotationas instantaneously.


%PL: This is all too technical, just require Java 7 for everything (I changed that too), and this needs no explenation.
%PreRender.jar is a multithreaded application, meaning that it performs best on any Java version > 7. 
%Older Java versions do not support our implementation of multithreading and should hence not be used.
PreRender.jar takes the following input files: %GB: are all of these mandatory or are some of these optional?
%PL: please change the command such that you only need to provide one directory, where all the subdirectories can be created (tree, clusters, csv, figtree).
%PL: This is much easier to use and to explain here.
\begin{itemize}
\item /path/to.phylogenetic.tree: Link to the phylogenetic tree used in the PhyloGeoTool.
\item /path/to/csvFile: Link to the csv file that connects nodes in the tool to attributes. Note: The id in the csv file has to be the same as the id of the nodes in the tree.
\item /path/to/distance\_matrix: Link to the distance matrix that was generated from this tree. % PL: refer to the previous section
\item /path/to/folder\_xml\_tree: Link to the folder where this jar can write its resulting tree files.
\item /path/to/folder\_xml\_clusters: Link to the folder where this jar can write its resulting cluster files.
\item /path/to/folder\_xml\_csv: Link to the folder where this jar can write its resulting csv files.
\item /path/to/folder\_figtree: Link to the folder where this jar can write its resulting figtree representations.
\end{itemize}

For example, the following command uses PreRender.jar to \ldots: %GB: add example command
%GB: and now what? Are we done here? Shall we show what the user should see now and what he/she can do next?
%PL: the web application needs to know which files to use right? is this configured in an XML?

\section{Load the visualization}



\end{document}
